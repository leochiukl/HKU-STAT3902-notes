\section{Empirical Probability Distributions}
\label{sect:empirical-prob-dist}
\begin{enumerate}
\item After studying \emph{probability theory} in STAT2901, we will focus on
\emph{statistical inference} in STAT3902: \emph{inference based on
observations}. In this section we will first focus on the \emph{observations} to
see how we can ``extract'' information from them.
\end{enumerate}
\subsection{Sample of Observations}
\begin{enumerate}
\item Much of statistical inference is based on the idea of a \emph{sample}.
Consider a ``population random variable'' \(X\) with cdf \(F(x)\). Then, a
(random) \defn{sample} of \(n\) (unrealized) observations from the population
is a vector of \(n\) \emph{random variables}: \((X_1,\dotsc,X_n)\) where
\[
X_1,\dotsc,X_n\iid F(x).
\]
\begin{center}
\begin{tikzpicture}
\node[draw, violet] () at (0,0.6) {data generating process};
\node[font=\large, violet] () at (0,0) {underpinning r.v.: \(X\)};
\draw[violet] (-4,1) rectangle (4,-1);
\foreach \x in {1,...,8} \draw[-Latex] (-4.5+\x,-0.5) -- (-4.5+\x,-1.5)
node[below=0.3cm, violet!80!white] {\(X_{\x}\)};
\node[violet] () at (-6,0) {population:};
\node[] () at (-6,-1) {generate \faIcon{copy}};
\node[violet!80!white] () at (-6,-2) {sample:};
\end{tikzpicture}
\end{center}
\begin{intuition}
Population can be understood as a \emph{data generating process}: Each
observation in the sample is a (random) datum.
\end{intuition}

\begin{note}
The \defn{sample size} of the sample \((X_1,\dotsc,X_n)\) is its number of
entries: \(n\).
\end{note}
\item A sample is \emph{unrealized} in the sense that it contains random
variables but not the actual observed \emph{values}. Depending on the outcome
\(\omega\in\Omega\), the actual observed values can differ in multiple
samplings (collections of samples).

In practice, we can only ``see'' \faIcon{eye} a \emph{realized} sample. However, for
theoretical purpose, we often work with \emph{unrealized} samples (consisting of
random variables).
\end{enumerate}

\subsection{Empirical Cumulative Distribution Function}
\begin{enumerate}
\item Given a sample \((X_1,\dotsc,X_n)\), the \defn{empirical cumulative
distribution function} (cdf) is
\[
\widehat{F}_n(x)=\frac{1}{n}\sum_{i=1}^{n}\indicset{X_i\le x}.
\]
\begin{remark}
\item In statistical inference, notation with a hat \(\;\widehat{}\;\) usually
\emph{estimates}/\emph{approximates} something.
\item In words, it means
\[
\widehat{F}_n(x)=\frac{\text{no.\ of observations \(\le\) \(x\)}}{\text{no.\ of observations}}.
\]
\end{remark}

\item Intuitively, the empirical cdf should be somewhat ``close'' to the actual
cdf, especially if the number of observations \(n\) is large. This is justified
by the following result:
\begin{theorem}[Glivenko-Cantelli theorem]
\label{thm:gilvenko-cantelli}
Suppose that we have \emph{infinitely many} (unrealized) observations
\(X_1,X_2,\dotsc\).  For any \(n\in\N\), the vector \((X_1,\dotsc,X_n)\)
constitutes a sample (of \(n\) observations), and we denote the corresponding
empirical cdf by \(\widehat{F}_n\). Then, the sequence of empirical cdfs
\(\{\widehat{F}_{n}\}\) \emph{converges uniformly} to the actual cdf \(F\)
almost surely (i.e., with probability 1).
\end{theorem}
\begin{note}
Roughly speaking, this suggests that \(\widehat{F}_{n}(x)\) is always within a
very small region around \(F(x)\) (``very close'') for any \(x\in\R\) (with
probability 1) when \(n\) is very large.
\end{note}
\end{enumerate}
